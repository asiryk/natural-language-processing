\phantomsection
\addcontentsline{toc}{section}{ВИСНОВКИ}

\pagestyle{empty}

\begin{center}
\textbf{\Large ВИСНОВКИ}
\end{center}

У ході виконання цієї роботи було розроблено інструмент, який дозволяє аналізувати
корпуси текстів на відповідні статистичні закономірності, такі як частота зустрічання
частин мови, частота універсальних залежностей від кореневого елементу речення, глибина
та ширина піддерева кожного слова, та інші.
Було розроблено механізм конвертації дерев до суфіксних дерев, що дає змогу
шукати спільні підрядки та їх частоти за заданою граматикою.

Цей інструмент не залежить від мови, тому можна аналізувати та
порівнювати будь-які мови корпусу. 
Додані інструменти по нормалізації корпусів сприяють тому, щоб різниця
у вхідних даних, при порівнянні різних мов була мінімальною.

Було проведено порівння корпусів з іспанською та українською мовами та встановлено кореляцію
деяких характеристик із законом Ципфа.

Перед виконанням цієї роботи я ознайомився з методами обробки природної мови
а також з науковими статистичними бібліотеками мови Python, а саме Pandas та Numpy, які використовувалися
в кінцевому програмному продукті.
