\phantomsection
\addcontentsline{toc}{section}{ВИСНОВКИ}

\pagestyle{empty}

\begin{center}
\textbf{\Large ВИСНОВКИ}
\end{center}


Проєкт Universal Dependencies спрямований на розробку міжмовно
узгоджених анотацій лінгвістичних корпусів для багатьох мов. Представлення
залежностей UD зосереджується на таких поняттях, як предикат, суб’єкт, об’єкт
речення, визначник іменника та модифікатор іменника. В результаті уточнення
граматичних категорій в UD сформовані теги так званої універсальної частини
мови (Universal Part of Speech).
Анотовані лінгвістичні корпуси UD представлені у форматі CoNLL-U, де
кожне слово характеризується, зокрема і синтетичними, показниками у десяти
тегах.
Розроблений програмний інструментарій дозволяє автоматизовано
аналізувати обраний корпус текстів і формувати сигнатури —-- набір формальних
показників та досліджувати їх.
Серед вилучених та побудованих формальних показників, --- синтаксична
роль (deprel) та універсальна частина мови (upos), синтаксично-морфологічний
показник (deprel+upos), сигнатура синтаксичних піддерев та повторювані підрядки
морфологічних рядків.
Встановлено, що із розглянутих показників тільки синтаксично-морфологічний
виявляє ознаки показника системності, що випливає із близькості
його розподілу до розподілу Ципфа, --- як у корпусі української, так і у корпусі
іспанської мови.
Повторювані підрядки морфологічних рядків речень ефективно
обчислюються з використанням побудованого на корпусі узагальненого суфіксного
дерева.

У ході виконання роботи були розглянуті основні методи по обробці речень та
використані наукові статистичні бібліотеки, написані мовою Python, а саме: Pandas
та Numpy. Було проведено модульне тестування окремих програмних компонентів
та застосовано розроблений інструментарій до корпусів UD з іспанською та
українською мовами.