\thispagestyle{empty}
\phantomsection  % make title clickable in toc
\addcontentsline{toc}{section}{ВСТУП}

\begin{center}
\textbf{\Large ВСТУП}
\end{center}

Обробка природної мови --- це область комп’ютерної лінгвістики, мета якої
полягає у знаходженні значимої інформації з текстів, подальшому аналізі цієї
інформації та прикладному застосуванні проаналізованих даних. Предметом
дослідження є природні мови --- ті, які природньо розвивалася з часом та які люди використовують для того, щоб комунікувати між собою.

Основоположною задачею при дослідження природних мов є створення класифікованих
та впорядкованих наборів даних, для того щоб з них можна було отримати
корисну інформацію, яку можна використати у прикладних сферах науки та програмування.
На основі таких даних можна створювати штучні інтелекти, для взаємодії з комп'ютером
використовуючи тільки природню мову. Прикладами таких систем є асистенти в популярних
мобільних операційних системах, які виводять інтерфейс користування пристроєм
на зовсім інший рівень. Але застосування подібних штучних інтелектів
не обмежується тільки голосовими асистентами. Дуже перспективними є варіанти застосування
для аналізу текстів та маркування їх за певними категоріями. Досконало дослідивши
закономірності в природних мовах можна значно підвищити точність машинного перекладу
(наприклад сервіс Google Translate) або створити системи, які перевіряють
тексти на граматичні та лексичні помилки (наприклад сервіс Grammarly). Такі системи
послуговуються не тільки набором вхідних слів, а й структурою речення,
його забарвленням та, навіть, контекстом, у якому воно вживається.

Метою цієї роботи є створення інструменту, який дозволяє досліджувати та порівнювати
статистичні закономірності у різних природніх мовах,
використовуючи корпуси текстів Universal Dependencies. Це дозволить краще розуміти
інформацію про структуру та форму даних, що будуть використовуватися для
створення моделей штучних інтелектів. Як наслідок, можна буде використовувити
найбільш підходящі набори даних, для навчання таких моделей.