\thispagestyle{empty}
\phantomsection  % make title clickable in toc
\addcontentsline{toc}{section}{ВСТУП}

\begin{center}
\textbf{\Large ВСТУП}
\end{center}

Обробка природної мови --- це область комп’ютерної лінгвістики, мета якої
полягає у знаходженні значимої інформації з текстів, подальшому аналізі цієї
інформації та прикладному застосуванні проаналізованих даних. Предметом
дослідження є природні мови --- ті, які природньо розвивалася з часом та які люди
використовують для того, щоб комунікувати між собою.

Головною задачею при дослідження природних мов є створення класифікованих
та впорядкованих наборів даних, для того щоб з них можна було отримати
корисну інформацію, з метою використання у прикладних сферах науки та програмування.
Серед найбільш популярних варіантів застосувань є розробка моделей штучних
інтелектів, які можуть слугувати для різноманітних цілей: створення мобільних
асистентів для керування пристроями за допомогою голосу, підвищення точності машинного перекладу (наприклад сервіс Google Translate), для аналізу текстів на
граматичні та лексичні помилки (наприклад сервіс Grammarly), для розбору текстів
та маркування їх за певними категоріями, контекстом та емоційним забарвленням.

\textbf{Мета роботи} полягає у створенні програмного інструментарію для
автоматизованого аналізу дерев універсальних залежностей та виявлення
характерних сигнатур змістовних одиниць коропусу текстів.
Для досягнення поставленої мети необхідно виконати \textbf{завдання}:

\begin{enumerate}
    \item Ознайомитись із основними засадами проєкту Universal Dependencies
    (UD), спрямованого на розробку міжмовно узгоджених корпусів текстів.
    \item Розглянути методи формального розбору та представлення речень.
    \item Розробити програмний інструментарій для автоматичної обробки корпусів
    UD.
    \item Застосувати розроблений інструментарій для аналізу вибраних корпусів UD.
\end{enumerate}