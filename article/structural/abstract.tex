\thispagestyle{empty}
\begin{center}
\textbf{\Large АНОТАЦІЯ}
\end{center}

Метою дипломної роботи є розробка програмного продукту для статистичного
аналізу формальних структур корпусу текстів Universal Dependencies.
Результат роботи програми може бути використаний дослідниками в області
штучного інтелекту та науки про дані, для кращого розуміння вибірки даних
для своїх моделей. 

Ця система була написана об’єктно-орієнтованою мовою програмування Python.
Виведення та візуалізація інформації відбувається у вигляді графіків, за
допомогою бібліотеки MatPlotLib. Для побудови моделі, структур даних
та статистичного аналізу використовувалися бібліотеки Pandas та Numpy.

Записка містить \pageref{LastPage} сторінок, 18 рисунків, 1 таблицю, 1 додаток і 21 посилання.

Ключові слова: python, обробка природної мови, аналіз даних, інтерпретована мова
програмування, візуалізація, статистичний аналіз.

\newpage
\thispagestyle{empty}

\begin{center}
\textbf{\Large ABSTRACT}
\end{center}

The purpose of the thesis is to develop a software product for statistical
analysis of the formal structures of the text corpora Universal Dependencies.
The results of the program can be used by researchers in the field of
artificial intelligence and data science, for a better understanding of
the data they going to use for their models.

The code was written in Python, an object-oriented programming language.
Output and visualization of the data are made via charts, using the MatPlotLib
library. Pandas and Numpy libraries were used to build a model, data structure
and to perform statistical analysis.

The thesis has a volume of \pageref{LastPage} sheets and contains 1 appendix and 21
references. There are also 18 figures and 1 table.

Keywords: python, natural language processing, data science, interpreted
programming language, visualization, statistical analysis.