\section{КОРПУСИ ТЕКСТІВ UNIVERSAL DEPENDENCIES}

% https://applied-language-technology.mooc.fi/html/notebooks/part_iii/02_universal_dependencies.html

% https://universaldependencies.org/u/pos/

\subsection{Корпуси текстів Universal Dependencies}
укр - 122275 токенів 122324 слова 7060 реченнь

Корпуси текстів Universal Dependencies або скорочено UD — це проект спільної розробки
з відкритим кодом, який має дві основні цілі:

\begin{enumerate}
    \item розробити загальну структуру для опису граматичної структури різноманітних мов.
    \cite{bib1}
    \item створювати анотовані корпуси або \hyperlink{term0}{трібанки} для різних мов,
    які застосовують цю структуру. \cite{bib2}
\end{enumerate}

Таким чином, проект має на меті забезпечити систематичний опис синтаксичних структур і
морфологічних особливостей у різних мовах, що, у свою чергу, дозволяє проводити порівняння
між мовами.

Лінгвістичні корпуси, які містять анотації для синтаксичних відносин,
часто називають банками дерев (трібанками), оскільки синтаксичні структури, як правило,
представлені за допомогою деревоподібних структур. Таким чином, у цьому контексті трібанк —
це просто набір синтаксичних дерев, які були послідовно анотовані за допомогою UD
або якоїсь іншої структури.

Структура UD є компромісом між кількома конкуруючими критеріями, які наведені нижче:

\begin{enumerate}
    \item UD має бути придатним для комп'ютерного розбору з високою точністю
    \item UD має бути задовільним для лінгвістичного аналізу окремих мов
    \item UD повинен гарно виявляти структурні подібності між спорідненими мовами
    \item UD має бути придатним для швидкого, послідовного анотування людиною-анотатором
    \item UD має бути легко зрозумілим і використаним користувачами, які не знають мови
\end{enumerate}


\subsection{Формат даних CoNLL-U}

\begin{table}[ht]
\begin{tabular}{|l|l|}
\hline
  Field   & Description \\ \hline
ID        & Index of the word in sequence \\ \hline
FORM      & The form of a word or punctuation symbol \\ \hline
LEMMA     & Lemma or the base form of a word \\ \hline
UPOS      & Universal part-of-speech tag \\ \hline
XPOS      & Language-specific part-of-speech tag \\ \hline
FEATS     & Morphological features \\ \hline
HEAD      & Syntactic head of the current word \\ \hline
DEPREL    & Universal dependency relation to the \\ \hline
DEPS      & Enhanced dependency relations \\ \hline
MISC      & Any additional annotations \\ \hline
\end{tabular}
\caption{Table to test captions and labels.}
\label{table:1}
\end{table}