\section{КОРПУСИ ТЕКСТІВ UNIVERSAL DEPENDENCIES}

Проблема відмінності природних мов заважає класифікації та порівнянню цих мов,
тому проєкт Universal Dependencies ставить на меті її вирішити~\cite{bib18}.

\subsection{Корпуси текстів Universal Dependencies}
Корпуси текстів Universal Dependencies або скорочено UD — це проєкт
спільної розробки з відкритим кодом, який має дві основні цілі:

\begin{enumerate}
    \item розробити загальну структуру для опису граматичної структури
    різноманітних мов~\cite{bib1}.
    \item створювати анотовані корпуси або \hyperlink{term0}{банки дерев} для
    різних мов,
    які застосовують цю структуру~\cite{bib2}.
\end{enumerate}

Таким чином, проєкт має на меті забезпечити систематичний опис синтаксичних структур і
морфологічних особливостей у різних мовах, що, у свою чергу, дозволяє проводити порівняння
між мовами. Це, в свою чергу значно полегшить роботу перекладачів
адже всі важливі характеристики мов будуть класифікованими по категоріям.

Лінгвістичні корпуси, які містять анотації для синтаксичних відносин,
часто називають банками дерев (трібанками), оскільки синтаксичні структури, як правило,
представлені за допомогою деревоподібних структур. Таким чином, у цьому контексті трібанк —
це просто набір синтаксичних дерев, які були послідовно анотовані за допомогою UD
або якоїсь іншої структури.

Структура UD є компромісом між кількома конкуруючими критеріями, які наведені нижче:

\begin{enumerate}
    \item UD має бути задовільним для лінгвістичного аналізу окремих мов
    \item UD має добре підходити для лінгвістичної типології, тобто 
    забезпечувати відповідну основу для виявлення міжмовного паралелізму між
    мовами та мовними сім’ями.
    \item UD має бути придатним для швидкого, послідовного анотування людиною-анотатором
    \item UD має бути придатним для комп'ютерного розбору з високою точністю
    \item UD повинен гарно виявляти структурні подібності між спорідненими мовами
    \item UD має бути легко зрозумілим і використаним користувачами, які не знають мови
\end{enumerate}

Попередником проєкту Universal Dependencies був Stanford dependencies, розроблений
у 2005 році як рушій для парсеру Stanford parser, щоб допомогти системам
розпізнавання тексту Recognizing Textual Entailment, після чого став стандартом
для аналізу залежностей англійської мови і з тих пір застосовувався до багатьох
мов~\cite{bib9}. Універсальний набір тегів Google виник у результаті міжмовного
аналізу помилок на основі співпраці CoNLL-X McDonald і Nivre (2007)~\cite{bib10},
вперше використаного для створення моделі, основаної на підході навчання без
учителя, яка вміла визначати тип універсальної частини мови. Після чого ця система
стала стандартом де-факто для відображення різноманітних наборів тегів. Наступною
ітерацією став інструмент Interset для перетворення між наборами тегів
морфологічного синтаксису різними мовами. Пізніше він був використаний як
морфологічний шар у проєкті HamleDT, для поєднання багатьох мов у загальній
структурі анотацій.

Проєкт Universal Dependency Treebank (UDT), в якому було створено банки дерев для
6 мов, був першою спробою поєднати, універсальні теги Google та Stanford
dependencies. Що послугувало приводом для створення проєкту універсальних
залежностей Стенфорда (USD). Фінальною ітерацією, яка закріпилася на момент
написання цієї роботи є проєкт Universal Dependencies, який є результатом
об’єднання всіх цих ініціатив в єдину когерентну структуру, засновану на
Універсальних залежностях Стенфорда, розширеній версії набору універсальних тегів
Google, переглянутої підмножини інвентаризації функцій Interset та переглянутої
версії формату CoNLL-X, що отримав назву CoNLL-U.

Перша версія Universal Dependencies, випущена ще в жовтні 2014 року, представила
дещо розширений набір мовних тегів загального призначення. Цей набір має деякі
відмінності від вихідного речення, але робить його менш насиченим і уточнює
визначення категорій. В результаті цієї роботи теги універсальної частини мови
(Universal Part of Speech) мають змістовні визначення і не обов'язково є
еквівалентними класами категорій у головному банку дерев для окремих мов. Тому
перетворення на теги універсальної частини часто вимагає змін контексту або
внесенння змін вручну. Морфологічні ознаки UD призначені для аналізу, щоб отримати
стислий набір основних ознак, які є унікальними та поширеними в усіх мовах.
Представлення залежностей UD розвивається зі Стенфордських залежностей, які самі
слідують ідеї опису, орієнтованого на граматичне відношення, який можна знайти в
багатьох мовних структурах. Він централізовано організований навколо таких понять,
як суб’єкт, об’єкт речення, визначник іменника та модифікатор іменника. Метою
Universal Dependencies є краща адаптація граматичних структур типологічно різних
мов, а також додавання чи покращення зв’язків, щоб очистити більш химерні та більш
специфічні для англійської мови особливості оригінальної версії. Таким чином, нова
таксономія має менше зв’язків, ніж оригінальна Stanford dependencies.

\subsection{Формат даних CoNLL-U}
Формат CoNLL-U є наступною ітерацією формату CoNLL-X~\cite{bib11},
що в свою чергу є схемою анотування лінгвістичних корпусів.
По суті, це звичайний текстовий файл, розмічений таким чином, щоб
його було зручно як і переглядати живій людині, та знаходити потрібні
частини мови та слова у реченнях, так і комп'ютеру, щоб незалежно
від мови програмування, на виході був максимально уніфікований формат даних.
Головною вимогою до цього формату є те, що текст закодований в ньому за
стандартом UTF-8 адже в різних мовах зустрічаються символи, які
неможливо відобразити, використовуючи таблицю ASCII. Зважаючи на
кількість та різноманітність мовних корпісів, анотованих в Universal Dependencies,
це є вимушенинм кроком, без якого існування такого проєкту є неможливим.
У цьому форматі є всього 3 типи рядків:

\begin{enumerate}
    \item Рядки слів, що містять анотацію слова/токена в 10 полях, розділених
    одиночними символами табуляції.
    \item Порожні рядки, що позначають межі речень.
    \item Коментарі, або рядки, які починаються з символа ``\#''.
\end{enumerate}

У більшості випадків текстові файли редагуються людьми вручну. Файли
CoNLL-U ж є форматом для обміну даними. Анотатори виконнують свою
роботу по розмітці текстів та визначенню універсальних частин мови
у графічному інтерфейсі, де основними елементами є дуги зв'язків
та мітки про тип та частину мову. Вони не вносять вручну даних до
цих файлів. У таких графічних прогрмах є спеціальні модулі,
які відповідають серіалізацію та експорт цих даних то файлів CoNLL-U.

Найважливіші дані для зберігаються у рядках слів. Тут знаходяться всі властивості,
проанотовані лінгвістами, повний перелік та опис яких зображено у
таблиці~\ref{table:conllu}

\begin{table}[ht]
\caption{Складові токенів слів у файлі CoNLL-U}
\label{table:conllu}
\centering
\begin{tabular}{|l|l|}
\hline
\multicolumn{1}{|c|}{Поле} & \multicolumn{1}{|c|}{Опис} \\ \hline
  
ID        & Індекс слова у реченні \\ \hline
FORM      & Форма слова або пунктуаційного знака \\ \hline
LEMMA     & Лема або базова форма слова \\ \hline
UPOS      & Тег універсальної частинини мови \\ \hline
XPOS      & Тег неуніверсальної частинини мови \\ \hline
FEATS     & Морфологічні особливості \\ \hline
HEAD      & Синтаксичний корінь поточного слова \\ \hline
DEPREL    & Універсальне відношення залежності до HEAD \\ \hline
DEPS      & Покращені відношення залежності \\ \hline
MISC      & Будь-які додаткові примітки \\ \hline
\end{tabular}
\end{table}

\subsection*{Висновки до розділу \arabic{section}}
\addcontentsline{toc}{subsection}{Висновки до розділу \arabic{section}}
У цьому розділі було оглянуто проєкт Universal Dependencies, його мету й
історію. Проведено порівняння між аналогічними проєктами, які існували до UD.
Також було розібрано формат даних CoNLL\nobreakdash-U.
