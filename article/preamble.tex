%%% Шрифти
\usepackage[english,ukrainian]{babel}
    \addto\captionsukrainian{\renewcommand{\figurename}{Рисунок}}
\setlength\parindent{1.25cm} % Абзацний відступ
\usepackage{fontspec}
    \setmainfont{Times New Roman}
    \setsansfont{Arial}
    \setmonofont{Fira Code}
\usepackage{setspace} % Міжрядковий інтервал
    \onehalfspacing % 1.5


%%% Береги та колонтитули
\usepackage{geometry}
    \geometry{top=25mm}
    \geometry{bottom=25mm}
    \geometry{left=25mm}
    \geometry{right=10mm}
\usepackage{fancyhdr}
    \pagestyle{fancy}
    \fancyhf{}
    \rfoot{\thepage}
    \renewcommand{\headrulewidth}{0mm}


%%% Математика
\usepackage{amsmath,amsfonts,amssymb,amsthm,mathtools}
\usepackage{euscript}
\usepackage{mathrsfs}


%%% Работа з картинками
\usepackage{graphicx}
\graphicspath{{images/}} % Шлях до директорій з картинками
\setlength\fboxsep{3pt} % Відступ рамки \fbox{} від картинки
\setlength\fboxrule{1pt} % Товщина ліній рамки \fbox{}
\usepackage{wrapfig} % Обтікання картинок і таблиць текстом


%%% Работа з таблицями
\usepackage{array,tabularx,tabulary,booktabs}
\usepackage{longtable}
\usepackage{multirow}


\usepackage{chngcntr} % Змінити лічильники на figure
    \counterwithin{figure}{section}
\usepackage[labelsep=endash]{caption}
    \DeclareCaptionLabelSeparator{endash}{ --- }
% \usepackage[labelsep=period,labelfont=bf,figurename={Хрень},figurewithin=none]{caption}
\usepackage{indentfirst} % Відступ у першомі параграфі секції
\usepackage[center]{titlesec} % Відцентрувати заголовки
\usepackage{hyperref}
    \hypersetup{
        colorlinks=true,
        citecolor=black,
        filecolor=black,
        linkcolor=black,
        urlcolor=black,
    }
    

%%% Команди
\let\oldsection\section
\renewcommand{\section}{\newpage\oldsection}